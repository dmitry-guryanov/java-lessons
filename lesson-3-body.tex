\section{Введение в объектно-ориентированное программирование часть 2}

\begin{frame}[fragile]
	\frametitle{\textit{this} -- это ссылка на текущий объект}

	\begin{large}
	Используется для того, чтобы:
	\begin{itemize}
	\item{Обратиться к полям, скрытым локальными переменными и параметрами методов}
	\item{Передать ссылку на текущий объект в метод другого объекта}
	\item{Вызвать другой конструктор из текущего класса}
	\end{itemize}
	\end{large}

	\begin{minted}[bgcolor=bgcode,baselinestretch=0.8,linenos=true]{java}
		class Rectangle {
		    float x1, y1, x2, y2;

		    Rectangle(float x1, float y1,
		            float x2, float y2) {
		        this.x1 = x1; this.y1 = y1;
		        this.x2 = x2; this.y2 = y2;
		    }
		
		    void testMethod(MyClass c) {
		        c.method(this);
		    }
		}
	\end{minted}
\end{frame}

\begin{frame}[fragile]
	\frametitle{\textit{this}}

	\begin{minted}[bgcolor=bgcode,baselinestretch=0.8,linenos=true]{java}
		class Button {
		    String label;
		    Window parent;

		    Button(Window parent, String label) {
		        this.parent = parent;
		        this.label = label;
		    }
		}

		class Window {

		    Button makeButton(String label) {
		        return new Button(this, label);
		    }
		}
	\end{minted}
\end{frame}

\begin{frame}[fragile]
	\frametitle{\textit{this}}

	\begin{minted}[bgcolor=bgcode,baselinestretch=0.8,linenos=true]{java}
		class Point {
		    int x, y;

		    Point(int x, int y) {
		        this.x = x;
		        this.y = y;
		    }

		    Point() {
		        this(0, 0);
		    }
		}
	\end{minted}
\end{frame}

\subsection{Строки}
\begin{frame}[fragile]
	\begin{Large}
	\frametitle{Строки в Java -- это объекты, а не массивы символов}

	Дополнительные возможности:
	\begin{itemize}
	\item{Создание как в языке С: \emph{\texttt{String s = "qwe";}}}
	\item{Доступен оператор \texttt{+}}
	\end{itemize}
	\medskip
	\begin{minted}[bgcolor=bgcode]{java}
	String s1 = "123";
	String s2 = "45";

	/* will print "12345" */
	System.out.println(s1 + s2);
	\end{minted}
	\end{Large}
\end{frame}

\begin{frame}[fragile]
	\frametitle{Полезные методы класса \textit{String}}

	\begin{large}
	Получить длину строки
	\begin{minted}[bgcolor=bgcode]{java}
	int length()
	\end{minted}

	Получить симовол в указанной позиции
	\begin{minted}[bgcolor=bgcode]{java}
	char charAt(int index)
	\end{minted}

	Сравнить строки (аналог функции \emph{strcmp} в С)
	\begin{minted}[bgcolor=bgcode]{java}
	int compareTo(String anotherString) 
	\end{minted}

	Сравнить строки (возвращает просто true или false)
	\begin{minted}[bgcolor=bgcode]{java}
	boolean equals(Object anObject)
	\end{minted}

	Сконвертировать в массив символов
	\begin{minted}[bgcolor=bgcode]{java}
	char[] toCharArray()
	\end{minted}
	\end{large}
\end{frame}

\begin{frame}[fragile]
	\frametitle{Полезные методы класса \textit{String}}

	\begin{large}
	Проверить, содержит ли данную последовательность символов
	\begin{minted}[bgcolor=bgcode]{java}
	boolean contains(CharSequence s)
	\end{minted}

	Разделить на части
	\begin{minted}[bgcolor=bgcode]{java}
	String[] split(String regex);
	\end{minted}

	Проверить, начинается ли строка с данной
	\begin{minted}[bgcolor=bgcode]{java}
	boolean startsWith(String prefix)
	\end{minted}

	\href{http://docs.oracle.com/javase/6/docs/api/java/lang/String.html}{документация по классу String}
	\end{large}
\end{frame}

\begin{frame}[fragile]
	\frametitle{Преобразование строк в число}

	\begin{large}
	Для преобразвания строк в числа можно воспользоваться классами \textit{Short}, \textit{Integer}, \textit{Long}, \textit{Float}, \textit{Double}.
	Пример:

	\begin{minted}[bgcolor=bgcode]{java}
	String s1 = "124";
	String s2 = "32.45"
	int x;
	float y;

	x = Integer.parseInt(s1);
	y = Float.parseFloat(s2);
	\end{minted}

	\end{large}
\end{frame}

\begin{frame}[fragile]
	\frametitle{Метод \textit{toString}}

	\begin{minted}[bgcolor=bgcode]{java}
		class Point {
		    int x, y;
		    Point(int x, int y) { this.x = x; this.y = y; }
		    public String toString() {
		        return String.format("Point(%d, %d)", x, y);
		    }
		}
		class Test {
		    public static void main(String []args) {
		        Point a = new Point(1,2);
		        Point b = new Point(4,2);

		        System.out.println(a);
		        System.out.printf("mypoint=%s\n", b);
		    };
		}
	\end{minted}

\end{frame}

\subsection{Задачи}
\begin{frame}[fragile]
	\frametitle{Задачи}

	\begin{enumerate}
		\begin{item}
			Написать программу для сортировки массива случайных чисел. Напечатать массив до и после сортировки.

			Дополнения:
			\begin{itemize}
				\item{Взять размер массива из параметра командной строки, если он не указан или указано более одного параметра - напечатать справку.}
				\item{Считать массив из параметров командной строки. Размер должен определяться из числа аргументов.}
			\end{itemize}
		\end{item}
		\begin{item}
			Найти все простые числа из диапазона, заданного в командной строке 
		\end{item}
		\begin{item}
			Написать класс Rectangle, добавить в него метод \textit{compareTo(Rectangle r)}, который сравнивает площать текущего
			с площадью прямоугольника \textit{r}, и метод \textit{toString()}.
			
			После этого сделать массив из случайных прямоугольников, отсортировать его и вывести на экран.

		\end{item}

	\end{enumerate}
\end{frame}

\subsection{Полезные ресурсы}
\begin{frame}
	\frametitle{Полезные ресурсы}

	\begin{itemize}
	\item{\href{http://docs.oracle.com/javase/6/docs/}{Официальная документация}}
	\item{\href{http://www.oracle.com/technetwork/java/javase/documentation/java-se-7-doc-download-435117.html}{Архив с официальной документацией}}
	\item{\href{http://docs.oracle.com/javase/tutorial/}{Java tutorials}}
	\item{\href{http://www.javable.com/tutorials/fesunov/}{Учебное пособие (на русском языке)}}
	\end{itemize}
\end{frame}

