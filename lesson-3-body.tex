

\begin{frame}[fragile]
	\frametitle{Конструкторы можно вызывать друг из друга}
	{\large Есть только одно ограничение -- вызов конструктора должен быть первый оператором.}

	\medskip
	\begin{minted}[bgcolor=bgcode,tabsize=4,linenos=true]{java}
		class Rectangle {
		    float x1, y1, x2, y2;

		    Rectangle(float ix1, float iy1,
		              float ix2, float iy2) {
		        x1 = ix1; y1 = iy1;
		        x2 = ix2; y2 = iy2;
		    }

		    Rectangle(float ix, float iy) {
		        Rectangle(0, 0, ix, iy);
		    }
		}
	\end{minted}

\end{frame}

\begin{frame}[fragile]
	\frametitle{Обычные методы тоже могут иметь одинаковые имена}

	\begin{large}
	\begin{minted}[bgcolor=bgcode,baselinestretch=0.8,linenos=true]{java}
		class Rectangle {
		    /* ... */
		    void scale(float sx, float sy) {
		        ix1 *= sx; ix2 *= sx;
		        iy1 *= sy; iy2 *= sy;
		    }

		    void scale(float s) {
		        ix1 *= s; ix2 *= s;
		        iy1 *= s; iy2 *= s;
		    }

		    /* better */
		    void scale(float s) {
		        scale(s, s);
		    }
		}
	\end{minted}
	\end{large}
\end{frame}

\begin{frame}[fragile]
	\frametitle{\textit{this} -- это ссылка на текущий объект}

	\begin{large}
	Используется для того, чтобы:
	\begin{itemize}
	\item{Передать ссылку на текущий объект в метод другого объекта}
	\item{Обратиться к полям, скрытым локальными переменными и параметрами методов}
	\end{itemize}
	\end{large}

	\begin{minted}[bgcolor=bgcode,baselinestretch=0.8,linenos=true]{java}
		class Rectangle {
		    float x1, y1, x2, y2;

		    Rectangle(float x1, float y1,
		            float x2, float y2) {
		        this.x1 = x1; this.y1 = y1;
		        this.x2 = x2; this.y2 = y2;
		    }
		
		    void testMethod(MyClass c) {
		        c.method(this);
		    }
		}
	\end{minted}
\end{frame}


\begin{frame}[fragile]
	\frametitle{Массивы}

	\begin{Large}
	Объявление ссылки на массив:
	\begin{minted}[bgcolor=bgcode]{java}
	int arr[];
	Rectangle rects[];
	\end{minted}

	Создание массива:
	\begin{minted}[bgcolor=bgcode]{java}
	arr = new int[20];
	\end{minted}

	Создание массива объектов:
	\begin{minted}[bgcolor=bgcode]{java}
	rects = new Rectangle[2];
	rects[0] = new Rectangle(1, 2);
	rects[1] = new Rectangle(1, 3);
	\end{minted}
	\end{Large}
\end{frame}

\begin{frame}[fragile]
	\frametitle{Массивы}

	\begin{Large}
	Создание и заполнение массива:
	\begin{minted}[bgcolor=bgcode]{java}
	int arr[] = {1, 2, 3};
	Rectangle rects = { new Rectangle(1, 2),
	                    new Rectangle(1, 3)};
	\end{minted}

	Длина массива:
	\begin{minted}[bgcolor=bgcode]{java}
	System.out.println(rects.length);
	\end{minted}

	Работа с массивом:
	\begin{minted}[bgcolor=bgcode]{java}
	x = arr[0] + arr[1];
	for(i = 0; i < arr.length; i++)
	    System.out.println(arr[i]);
	\end{minted}

	\end{Large}
\end{frame}

\begin{frame}[fragile]
	\frametitle{Строки в Java -- это объекты}

	Дополнительные возможности
\end{frame}

