\section{Пакеты и модификаторы доступа}
\subsection{Пакеты}

\begin{frame}[fragile]
	\frametitle{Библиотека Java — это сборник классов}

	\begin{Large}
	\begin{itemize}
	\item{Библиотеки нужны для того, чтобы использовать в своем проекте классы, написанные другими людьми или наоборот, написать классы, которые потом смогут использовать другие.}
	\item{Обычно проекты выполнены в виде библиотеки, в который каждый файл содержит один класс.}
	\end{itemize}
	\end{Large}
\end{frame}

\begin{frame}[fragile]
	\frametitle{Использование библиотек}

	\begin{large}
	Обычно физически библиотека это jar-файл (архив), но может быть и просто директорией. Для подключения библиотеки надо использовать параметр \emph{-cp} при компиляции и запуске или переменную окружения \emph{CLASSPATH}:
	\medskip

	\begin{minted}[bgcolor=bgcode]{bash}
	javac -cp /usr/share/java/protobuf.jar App.java
	javac -cp /home/student/mylib App.java
	
	export CLASSPATH=/home/student/mylib
	javac App.java
	\end{minted}
	\end{large}

	По-умолчанию подключена только системная библиотека Java.
\end{frame}

\begin{frame}[fragile]
	\frametitle{Библиотеки состоят из пакетов}

	Библиотека --- это очень большое хранилище классов, обычно она разделена на более мелкие --- пакеты.
	Например у нас есть следующая структура файлов на диске:
	\medskip
	\begin{columns}[c]
	\column{2.0in}
	\begin{verbatim}
mylib
   |-package1
       |-ClassA.java
       |-package2
          |-ClassB.java
	\end{verbatim}
	\medskip
	\begin{small}
	В файлах ClassA.java и ClassB.java должно быть указано имя пакета с помощью ключевого слова \emph{package}.
	\end{small}
	\column{2.45in}
	CLassA.java:
	\begin{minted}[bgcolor=bgcode]{java}
	package package1;

	class ClassA {
	};
	\end{minted}

	ClassB.java:
	\begin{minted}[bgcolor=bgcode]{java}
	package package1.package2;

	class ClassB {
	};
	\end{minted}
	\end{columns}
\end{frame}

\begin{frame}[fragile]
	\frametitle{Использование пакетов}
	\begin{large}
	Теперь мы можем подключить эту библиотеку и воспользоваться классами \emph{ClassA} и \emph{ClassB} из любого места:
	\medskip

	\begin{minted}[bgcolor=bgcode]{java}
	import package1.ClassA;
	import package1.package2.ClassB;

	class App {
	    ClassA a = new ClassA();
	    ClassB b = new ClassB();
	};
	\end{minted}

	Компиляция и запуск:
	\begin{minted}[bgcolor=bgcode]{bash}
	javac -cp /home/student/mylib App.java
	java -cp /home/student/mylib App
	\end{minted}
	\end{large}

\end{frame}

\begin{frame}[fragile]
	\frametitle{Различные варианты использования}

	\begin{large}
	Можно подключить один класс из пакета:
	\begin{minted}[bgcolor=bgcode]{java}
	import package1.ClassA;
	\end{minted}

	Все классы из пакета:
	\begin{minted}[bgcolor=bgcode]{java}
	import package1.*;
	\end{minted}

	Либо вообще не подключать с помощью \emph{import}, а каждый раз указывать полный путь:
	\begin{minted}[bgcolor=bgcode]{java}
	class App {
	    public static void main(String args[]) {
	        package1.ClassA a;
	        a = new package1.ClassA;
	    }
	}
	\end{minted}
	\end{large}
\end{frame}

\subsection{Модификаторы доступа}

\begin{frame}[fragile]
	\frametitle{Проблема с написанием классов для других}
	Допустим у нас был класс для описания линии:
	\begin{minted}[bgcolor=bgcode,linenos=true]{java}
	class Line {
	    int x1, y1, x2, y2;

	    Line(int x1, int y1, int x2, int y2) {
	        this.x1 = x1; this.y1 = y1;
	        this.x2 = x2; this.y2 = y2;
	    }

	    void draw(Graphics g) { ... };
	}
	\end{minted}

	Мы дали этот класс другу, который написал следующий код:
	\begin{minted}[bgcolor=bgcode,linenos=true]{java}
	Line l = new Line(10, 20, 30, 40);
	System.out.format("line: (%d-%d) - (%d-%d)",
	                  l.x1, l.y1, l.x2, l.y2);
	\end{minted}
\end{frame}

\begin{frame}[fragile]
	\frametitle{Проблема с написанием классов для других}
	Потом мы решили использовать новый класс Point для хранения координат:
	\begin{minted}[bgcolor=bgcode,linenos=true]{java}
	class Point {
	    int x, y;
	    Point(int x, int y) {
	        this.x = x; this.y = y
	    }
	}
	class Line {
	    Point p1, p2;
	    Line(int x1, int y1, int x2, int y2) {
	        p1 = new Point(x1, y1);
	        p2 = new Point(x2, y2);
	    }
	}
	\end{minted}

	После чего дали этот новый код другу. У него все сломалось.
\end{frame}

\begin{frame}[fragile]
	\frametitle{Что делать ?}
	\begin{Large}
	Надо описывать, что в нашем классе можно использовать и что нельзя. То, что можно использовать всем мы не должны менять ни в коем случае.\end{Large}
	
	\begin{small} (в этом случае придется обойти всех, кто воспользовался нашим классом, объяснить им, зачем мы сделали такое изменение, терпеливо выслушать нецензурную брань и уговорить их исправить свой код).\end{small}

	\medskip
	\begin{Large}
	Это можно сделать с помощью модификаторов доступа.
	\end{Large}
\end{frame}

\begin{frame}[fragile]
	\frametitle{Модификаторы доступа для классов}

	\begin{Large}
	\begin{itemize}
	\item{\emph{public} --- класс можно использовать везде}
	\item{\emph{без модификатора} - класс можно использовать только внутри пакета}
	\end{itemize}
	\end{Large}

	\begin{minted}[bgcolor=bgcode,linenos=true]{java}
	class Point {
	    ...
	};

	public class Line {
	    ....
	};
	\end{minted}

	\emph{public} классы обязательно должны быть описаны в файлах с таким же именем.
\end{frame}

\begin{frame}[fragile]
	\frametitle{Модификаторы доступа для членов класса}

	{\bf
	\begin{large}
	\begin{tabular}[b]{|l|c|c|c|c|}
	\hline
	&класс&пакет&подкласс&везде\\[2ex]
	\hline
	public&Y&Y&Y&Y\\[2ex]
	\hline
	protected&Y&Y&Y&X\\[2ex]
	\hline
	без модификатора&Y&Y&X&X\\[2ex]
	\hline
	private&Y&X&X&X\\[2ex]
	\hline
	\end{tabular}
	\end{large}
	}
\end{frame}

\begin{frame}[fragile]
	\frametitle{Пример}

	\emph{test/ClassA.java}:
	\begin{minted}[bgcolor=bgcode,linenos=true]{java}
	package test;
	class ClassA {
	    public int a;
	    int b;
	    protected int c;
	    private d;
	};
	\end{minted}

	\emph{test/ClassB.java}
	\begin{minted}[bgcolor=bgcode,linenos=true]{java}
	package test;
	class ClassB extends ClassA {
	    void method1() {
	        print(a + b); /* OK */
	        print(c); /* OK */
	        print(d); /* err */
	    };
	};
	\end{minted}
\end{frame}

\begin{frame}[fragile]
	\frametitle{Пример}
	\emph{Test.java}:
	\begin{minted}[bgcolor=bgcode,linenos=true]{java}
	import test.*;

	class Test {
	    void method2() {
	        ClassA x = new ClassA();
	        print(x.a); /* OK */
	        print(x.b); /* err */
	        print(x.c); /* err */
	        print(x.d); /* err */

	    }
	}
	\end{minted}

\end{frame}

\begin{frame}[fragile]
	\frametitle{Правила расстановки можификаторов доступа}
	\begin{enumerate}
	\begin{item}
	Поля делать \emph{private}. Eсли нужно получать доступ к полю извне --- написать методы для этого, например
	\begin{minted}[bgcolor=bgcode,linenos=true]{java}
	class ClassA {
	    private int x;
	    int getX() { return x; }
	    void setX(int x) { this.x = x; }
	}
	\end{minted}
	\end{item}
	\item{Если метод объявлен как \emph{public} --- нельзя менять его поведение, тип возвращаемого значения, аргументы. Также нельзя его удалять.}
	\item{Вспомогательные классы объявлять без модификатора.}

	\end{enumerate}
\end{frame}

\begin{frame}[fragile]
	\frametitle{Задача}
Написать класс для работы со стеком.
Сделать 3 класса: один класс, обекты которого будут храниться в стеке (например классы Point или Line из этого семинара), класс Node, в котором будет храниться указатель на элемент стека, и, например, указатели на предыдущий и следующий элементы. И класс Stack, который будет содержать методы push и pop.

Пример:

\begin{minted}[bgcolor=bgcode,linenos=true]{java}
Stack stack = new Stack();
Point p;

stack.push(new Point(1,2));
stack.push(new Point(2,3));

p = stack.pop();
print(p.getX());
\end{minted}

	
\end{frame}
