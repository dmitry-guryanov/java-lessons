\section{Введение в объектно-ориентированное программирование}

\subsection{Понятие объектно-ориентированного программирования}

\begin{frame}
	\frametitle{Определение ООП}

	\begin{large}
	\emph{Объектно-ориентированное программирование} -- парадигма программирования, в которой основными концепциями являются понятия объектов и классов. (©wikipedia)
	\end{large}
\end{frame}

\begin{frame}
	\frametitle{Объекты в реальном мире имеют состояние и поведение.}

	Объект: велосипед\\
	Состояние:
	\begin{itemize}
		\item{положение руля}
		\item{текущая передача}
		\item{скорость вращения педалей}
		\item{положение в пространстве}
	\end{itemize}
	Поведение:
	\begin{itemize}
		\item{сменить передачу}
		\item{повернуть руль}
		\item{узнать текущую скорость}
	\end{itemize}
\end{frame}

\begin{frame}[fragile]
	\frametitle{Объекты в программировании похожи на объекты из реального мира}

	\begin{columns}[c]
	\column{2.1in}
	\begin{large}
	Состояние -- набор полей, аналог полей структур в языке C.

	\medskip
	Поведение -- методы, аналог функций, применяемых к конкретному объекту.
	\end{large}
	\column{2.3in}
	\begin{minted}[bgcolor=bgcode]{java}
	class Bicycle {
	    /* fields */
	    float handlebarAngle;
	    int gear;
	    float cadence;
	    Vector pos;

	    /* methods */
	    void setGear(int gear);
	    void turn(float angle);
	    float getSpeed();
	}
	\end{minted}
	\end{columns}
\end{frame}

\begin{frame}
	\frametitle{Класс -- это тип объекта}
	\begin{large}
	В реальном мире много объектов с одинаковым набором свойств и поведением. Например много велосипедов одной модели.

	\medskip
	В программировании это называется классом. Каждый объект принадлежит какому-либо классу (является экземпляром этого класса).
	\end{large}
\end{frame}

\begin{frame}
	\frametitle{Свойства ООП}
	\begin{small}
	\begin{itemize}
		\item{Всё является объектом (в реальных языках часто это не так, например в Java есть небольшой набор простых типов данных, переменные этих типов не являются объектами).}
		\item{Вычисления осуществляются путём взаимодействия (обмена данными) между объектами, при котором один объект требует, чтобы другой объект выполнил некоторое действие. Объекты взаимодействуют, вызывая методы друг друга (аналоги функций в C)}
		\item{Каждый объект имеет независимую память, которая состоит из других объектов.}
		\item{Каждый объект является представителем класса, который выражает общие свойства объектов (таких, как целые числа или списки).}
		\item{В классе задаётся поведение (функциональность) объекта. Тем самым все объекты, которые являются экземплярами одного класса, могут выполнять одни и те же действия.}
	\end{itemize}
	\end{small}
\end{frame}

\begin{frame}[fragile]
	\frametitle{Сравним процедурное и объектно-ориентированное программирование}

	\begin{columns}[c]
	\column{2.1in}
	\emph{C-style}
	\medskip
	\begin{minted}[bgcolor=bgcode]{java}
	int fd;
	char buf[16];

	fd = open("file.txt");

	fseek(fd, 10);
	read(fd, buf, 16);

	close(fd);
	\end{minted}
	\column{2.1in}
	\emph{C++-style}
	\medskip
	\begin{minted}[bgcolor=bgcode]{java}
	File f;
	String s;

	f = new File("file.txt");

	f.fseek(10);
	s = f.read(16);

	delete f;
	\end{minted}
	\end{columns}
\end{frame}

\subsection{Язык Java}
\begin{frame}
	\frametitle{Java}

	\begin{Large}
	\emph{Java} — объектно-ориентированный язык программирования, разработанный компанией Sun Microsystems (в последующем приобретённой компанией Oracle). Дата официального выпуска — 23 мая 1995 года.
	\end{Large}
\end{frame}

\begin{frame}
	\frametitle{Java во многом отличается от C/C++}

	\begin{itemize}
		\item{Остутствуют возможности процедурного программирования (нет функций и структур в чистом виде)}
		\item{Остутствуют низкоуровневые возможности (работа с адресами скрыта, программист не должен делать никаких предположений о том, как передаются аргументы функций и.т.д.)}
		\item{Работа с памятью -- автоматическая, память объектов освобождается, когда не остается ссылок на них.}
		\item{Есть всего 8 простых типов данных, все остальные - классы}
		\item{Программа компилируется не в исполняемый файл, а в специальный байт-код, который затем может быть исполнен с помощью виртуальной машины java}
	\end{itemize}

\end{frame}


\begin{frame}[fragile]
	\frametitle{Hello World !}

	\begin{large}
	\emph{HelloWorldApp.java:}
	\begin{minted}[bgcolor=bgcode,gobble=0]{java}
	class HelloWorldApp {
	    public static void main(String[] args) {
	        System.out.println("Hello World!");
	    }
	}
	\end{minted}

	\emph{Компиляция:}
	\begin{minted}[bgcolor=bgcode,gobble=0]{bash}
	javac HelloWorldApp.java
	\end{minted}

	\emph{Запуск:}
	\begin{minted}[bgcolor=bgcode,gobble=0]{bash}
	java HelloWorldApp
	\end{minted}
	\end{large}

\end{frame}

\begin{frame}
	\frametitle{Простые типы данных}

	\begin{LARGE}
	\emph{byte, short, int, long, float, double, boolean, char}
	\end{LARGE}
\end{frame}

\subsection{Классы и объекты}
\begin{frame}[fragile]
	\frametitle{Посмотрим на пример описания и использования класса}
	\begin{minted}[bgcolor=bgcode,gobble=0,linenos=true,baselinestretch=0.7]{java}
	class Rectangle {
	    float x1, y1, x2, y2; 

	    Rectangle(float ix1, float iy1,
	              float ix2, float iy2) {
	        x1 = ix1; y1 = iy1;
	        x2 = ix2; y2 = iy2;
	    }   

	    float getSquare() {
	        return (x2 - x1) * (y2 - y1);
	    }   
	}

	class HelloWorldApp {
	    public static void main(String [] argc) {
	        Rectangle r = new Rectangle(1, 4, 7, 6);
	        float s = r.getSquare();
	        System.out.println(s);
	    }   
	}
	\end{minted}
\end{frame}

\begin{frame}[fragile]
	\frametitle{Использование объектов}

	\begin{minted}[bgcolor=bgcode,gobble=0,linenos=true]{java}
	class HelloWorldApp {
	    public static void main(String [] argc) {
	        Rectangle r1 = new Rectangle(1, 4, 7, 6);
	        Rectangle r2 = new Rectangle(0, 0, 5, 4);
	        float s = r1.getSquare();

	        System.out.println(s);
	        s = r2.getSquare();
	        System.out.println(s);
	    }
	}
	\end{minted}
\end{frame}


\begin{frame}[fragile]
	\frametitle{Статические поля и методы}

	\begin{large}
	Поля и методы, описанный с ключевым словом \emph{static} относятся не к объекту, а к классу.
	\end{large}

	\medskip
	\begin{minted}[bgcolor=bgcode,gobble=0,linenos=true,baselinestretch=1.0]{java}
	class Test {
	    static int x;
	    int y;

	    static void test1() {
	        System.out.println(x);
	    }

	    static void test2() {
	        System.out.println(y); /* error */
	    }
	}
	\end{minted}
\end{frame}

\begin{frame}[fragile]
	\frametitle{Статические поля и методы}
	\begin{minted}[bgcolor=bgcode,gobble=0,linenos=true,baselinestretch=1.0,firstnumber=13]{java}
	class HelloWorldApp {
	    public static void main(String [] argc) {
	        Test t1 = new Test();
	        Test t2 = new Test();
	        t1.x = 10;

	        System.out.println(t2.x);
	        t2.test1();

	        System.out.println(Test.x);
	        Test.test1();
	    }
	}
	\end{minted}

\end{frame}

\subsection{Описание классов}
\begin{frame}[fragile]
	\frametitle{Конструкторы можно вызывать друг из друга}
	{\large Есть только одно ограничение -- вызов конструктора должен быть первый оператором.}

	\medskip
	\begin{minted}[bgcolor=bgcode,tabsize=4,linenos=true]{java}
		class Rectangle {
		    float x1, y1, x2, y2;

		    Rectangle(float ix1, float iy1,
		              float ix2, float iy2) {
		        x1 = ix1; y1 = iy1;
		        x2 = ix2; y2 = iy2;
		    }

		    Rectangle(float ix, float iy) {
		        Rectangle(0, 0, ix, iy);
		    }
		}
	\end{minted}

\end{frame}

\begin{frame}[fragile]
	\frametitle{Обычные методы тоже могут иметь одинаковые имена}

	\begin{large}
	\begin{minted}[bgcolor=bgcode,baselinestretch=0.8,linenos=true]{java}
		class Rectangle {
		    /* ... */
		    void scale(float sx, float sy) {
		        ix1 *= sx; ix2 *= sx;
		        iy1 *= sy; iy2 *= sy;
		    }

		    void scale(float s) {
		        ix1 *= s; ix2 *= s;
		        iy1 *= s; iy2 *= s;
		    }

		    /* better */
		    void scale(float s) {
		        scale(s, s);
		    }
		}
	\end{minted}
	\end{large}
\end{frame}


\subsection{Массивы}
\begin{frame}[fragile]
	\frametitle{Массивы}

	\begin{Large}
	Объявление ссылки на массив:
	\begin{minted}[bgcolor=bgcode]{java}
	int arr[];
	Rectangle rects[];
	\end{minted}

	Создание массива:
	\begin{minted}[bgcolor=bgcode]{java}
	arr = new int[20];
	\end{minted}

	Создание массива объектов:
	\begin{minted}[bgcolor=bgcode]{java}
	rects = new Rectangle[2];
	rects[0] = new Rectangle(1, 2);
	rects[1] = new Rectangle(1, 3);
	\end{minted}
	\end{Large}
\end{frame}

\begin{frame}[fragile]
	\frametitle{Массивы}

	\begin{Large}
	Создание и заполнение массива:
	\begin{minted}[bgcolor=bgcode]{java}
	int arr[] = {1, 2, 3};
	Rectangle rects = { new Rectangle(1, 2),
	                    new Rectangle(1, 3)};
	\end{minted}

	Длина массива:
	\begin{minted}[bgcolor=bgcode]{java}
	System.out.println(rects.length);
	\end{minted}

	Работа с массивом:
	\begin{minted}[bgcolor=bgcode]{java}
	x = arr[0] + arr[1];
	for(i = 0; i < arr.length; i++)
	    System.out.println(arr[i]);
	\end{minted}

	\end{Large}
\end{frame}
